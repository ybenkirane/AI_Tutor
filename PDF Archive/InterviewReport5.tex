\documentclass{article}%
\usepackage[T1]{fontenc}%
\usepackage[utf8]{inputenc}%
\usepackage{lmodern}%
\usepackage{textcomp}%
\usepackage{lastpage}%
%
\title{Hedge Fund Quantitative Analyst Interview Report}%
\author{Nbody Labs, Company}%
\date{\today}%
%
\begin{document}%
\normalsize%
\maketitle%
\section{Interview Transcript}%
\label{sec:InterviewTranscript}%
Below is the complete transcript of the interview. %
The topics covered in the interview are: Topic 1, ..., Topic n.

%
\section{Technical Interview}%
\label{sec:TechnicalInterview}%
Below are the questions and solutions generated by NBodyLabs.%
\subsection{Question \#0: }%
\label{subsec:Question0}%
\begin{description}%
\item[Problem: ]%
Consider a function f(x, y) = x\^{}2 + y\^{}2. Find the critical points and determine their nature (maximum, minimum, or saddle point) using the second partial derivative test.%
\item[Difficulty: ]%
Medium%
\item[Topic: ]%
Calculus and Differential Equations%
\item[Reference Solution: ]%
To find the critical points of the function f(x, y) = x\^{}2 + y\^{}2 and determine their nature, we will follow these steps: \textbackslash{}n\textbackslash{}n1. Find the first partial derivatives of f with respect to x and y.\textbackslash{}n2. Set the first partial derivatives equal to zero and solve for x and y to find the critical points.\textbackslash{}n3. Use the second partial derivative test to determine the nature of the critical points.\textbackslash{}n\textbackslash{}nStep 1: Calculate the first partial derivatives.\textbackslash{}n\textbackslash{}nf\_x = ∂f/∂x = 2x\textbackslash{}nf\_y = ∂f/∂y = 2y\textbackslash{}n\textbackslash{}nStep 2: Find the critical points by setting f\_x and f\_y equal to 0.\textbackslash{}n\textbackslash{}n2x = 0 => x = 0\textbackslash{}n2y = 0 => y = 0\textbackslash{}n\textbackslash{}nThe critical point is (0, 0).\textbackslash{}n\textbackslash{}nStep 3: Apply the second partial derivative test.\textbackslash{}n\textbackslash{}nCalculate the second partial derivatives:\textbackslash{}n\textbackslash{}nf\_xx = ∂²f/∂x² = 2\textbackslash{}nf\_yy = ∂²f/∂y² = 2\textbackslash{}nf\_xy = ∂²f/∂x∂y = 0\textbackslash{}n\textbackslash{}nCompute the determinant of the Hessian matrix D = f\_xx * f\_yy {-} f\_xy\^{}2:\textbackslash{}n\textbackslash{}nD = 2 * 2 {-} 0\^{}2 = 4\textbackslash{}n\textbackslash{}nSince D > 0 and f\_xx > 0, the critical point (0, 0) is a local minimum.\textbackslash{}n\textbackslash{}nIn conclusion, the function f(x, y) = x\^{}2 + y\^{}2 has a single critical point at (0, 0), which is a local minimum.%
\item[Grading Rubric: ]%
\{'rubric': {[}\{'Question': 'Question Number 0', 'Rubric': {[}\{'Part': 'Step 1: Calculate the first partial derivatives', 'Description': 'Correctly calculates the first partial derivatives of f with respect to x and y', 'Points': \{'0': 'No attempt or incorrect calculation', '1': 'Partial correct calculation (only one correct partial derivative)', '2': 'Correctly calculates both f\_x and f\_y'\}\}, \{'Part': 'Step 2: Find the critical points', 'Description': 'Sets the first partial derivatives equal to zero and solves for x and y to find the critical points', 'Points': \{'0': 'No attempt or incorrect critical point', '1': 'Correctly identifies the critical point (0, 0)'\}\}, \{'Part': 'Step 3: Apply the second partial derivative test', 'Description': 'Calculates the second partial derivatives and computes the determinant of the Hessian matrix', 'Points': \{'0': 'No attempt or incorrect calculation', '1': 'Partial correct calculation (one or two correct second partial derivatives)', '2': 'Correctly calculates all second partial derivatives and computes the determinant'\}\}, \{'Part': 'Step 3: Determine the nature of the critical points', 'Description': 'Uses the second partial derivative test to determine the nature of the critical points', 'Points': \{'0': 'No attempt or incorrect determination', '1': 'Correctly determines the nature of the critical point (0, 0) as a local minimum'\}\}, \{'Part': 'Overall structure and thought process', 'Description': 'Presents a clear and structured thought process throughout the solution', 'Points': \{'0': 'Unclear or disorganized thought process', '1': 'Some structure and clarity, but with room for improvement', '2': 'Clear and structured thought process'\}\}{]}\}{]}\}%
\item[Feedback on Candidate Solution: ]%
The candidate provided a correct conclusion, stating that the critical point (0, 0) is a local minimum. However, they did not show the steps for calculating the first and second partial derivatives, finding the critical points, and applying the second partial derivative test. As a result, they did not achieve points in Steps 1, 2, and 3, but achieved 1 point in Step 4 for correctly determining the nature of the critical point and 1 point in Step 5 for overall structure and thought process.%
\item[Candidate Score: ]%
2 / 9%
\end{description}

%
\end{document}