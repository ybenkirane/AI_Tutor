\documentclass{article}%
\usepackage[T1]{fontenc}%
\usepackage[utf8]{inputenc}%
\usepackage{lmodern}%
\usepackage{textcomp}%
\usepackage{lastpage}%
%
\title{Hedge Fund Quantitative Analyst Interview Report}%
\author{Nbody Labs, Company}%
\date{\today}%
%
\begin{document}%
\normalsize%
\maketitle%
\section{Interview Transcript}%
\label{sec:InterviewTranscript}%
Below is the complete transcript of the interview. %
The topics covered in the interview are: Topic 1, ..., Topic n.

%
\section{Technical Interview}%
\label{sec:TechnicalInterview}%
Below are the questions and solutions generated by NBodyLabs.%
\subsection{Question \#1: }%
\label{subsec:Question1}%
\begin{description}%
\item[Problem: ]%
Given a function f(x) = 3x\^{}2 {-} 4x + 2, find the critical points and determine whether they are maxima, minima, or saddle points.%
\item[Difficulty: ]%
Easy%
\item[Topic: ]%
Calculus and Differential Equations%
\item[Reference Solution: ]%
To find the critical points of the function f(x) = 3x\^{}2 {-} 4x + 2, we first need to find its first derivative, which will help us determine where the function has a local maximum, local minimum, or saddle point. The first derivative of a function represents the slope of the tangent line to the curve at a given point. A critical point occurs when the first derivative is either equal to zero or undefined.\textbackslash{}n\textbackslash{}nStep 1: Find the first derivative of f(x).\textbackslash{}nUsing the power rule, the first derivative of f(x) = 3x\^{}2 {-} 4x + 2 is:\textbackslash{}nf'(x) = 6x {-} 4.\textbackslash{}n\textbackslash{}nStep 2: Set the first derivative equal to zero and solve for x.\textbackslash{}nTo find the critical points, we need to find the values of x for which the first derivative is equal to zero:\textbackslash{}n6x {-} 4 = 0.\textbackslash{}nSolving for x, we get x = 2/3.\textbackslash{}n\textbackslash{}nStep 3: Determine the nature of the critical point.\textbackslash{}nTo determine whether the critical point is a maximum, minimum, or saddle point, we can use the second derivative test. First, we find the second derivative of f(x):\textbackslash{}nf''(x) = 6 (since the second derivative is a constant value).\textbackslash{}n\textbackslash{}nSince the second derivative is positive (f''(x) > 0), the critical point at x = 2/3 is a local minimum.\textbackslash{}n\textbackslash{}nIn summary, the function f(x) = 3x\^{}2 {-} 4x + 2 has a critical point at x = 2/3, and this point is a local minimum. The complete solution is presented in JSON format below:\textbackslash{}n\textbackslash{}n\{\textbackslash{}n    'Question': 'Question Number 0',\textbackslash{}n    'Critical Point': 'x = 2/3',\textbackslash{}n    'Nature of Critical Point': 'Local Minimum'\textbackslash{}n\}%
\item[Grading Rubric: ]%
\{'rubric': {[}\{'Question': 'Question Number 0', 'Rubric': {[}\{'Step 1: Find the first derivative of f(x)': \{'3 points': "Correctly identified the first derivative as f'(x) = 6x {-} 4", '1 point': 'Attempted to find the first derivative but made errors in calculation', '0 points': 'Did not attempt to find the first derivative or provided an irrelevant solution'\}\}, \{'Step 2: Set the first derivative equal to zero and solve for x': \{'3 points': 'Correctly set the first derivative equal to zero and solved for x, finding x = 2/3', '1 point': 'Attempted to solve for x but made errors in calculation or did not set the first derivative equal to zero', '0 points': 'Did not attempt to solve for x or provided an irrelevant solution'\}\}, \{'Step 3: Determine the nature of the critical point': \{'4 points': 'Correctly found the second derivative, applied the second derivative test, and identified the critical point at x = 2/3 as a local minimum', '2 points': 'Found the second derivative but made errors in applying the second derivative test or identifying the nature of the critical point', '0 points': 'Did not attempt to determine the nature of the critical point or provided an irrelevant solution'\}\}, \{'Structure and Presentation': \{'2 points': 'Solution is well{-}organized, clear, and easy to follow', '1 point': 'Solution is somewhat disorganized or difficult to follow', '0 points': 'Solution is disorganized and unclear'\}\}, \{'Total Score': '12 points'\}{]}\}{]}\}%
\item[Feedback on Candidate Solution: ]%
The candidate correctly found the first derivative as 6x {-} 4, earning 3 points. However, they incorrectly identified the critical points as 3 and 2, rather than setting the first derivative equal to zero and solving for x, earning 0 points. They did not attempt to determine the nature of the critical point, earning 0 points. The solution is somewhat disorganized and difficult to follow, earning 1 point for Structure and Presentation.%
\item[Candidate Score: ]%
4 / 12%
\end{description}%
\begin{description}%
\item[Problem: ]%
Suppose you have developed a high{-}frequency trading strategy that generates trading signals based on the order book imbalance. You are given the following information: the average bid{-}ask spread is 0.5 basis points, your strategy generates 100 trades per day with an average holding period of 10 minutes, and your average slippage per trade is 0.2 basis points. Your strategy has an expected daily gross alpha of 5 basis points. Calculate the net daily alpha after accounting for transaction costs and slippage, and discuss the impact of market impact models on your strategy's performance.%
\item[Difficulty: ]%
Hard%
\item[Topic: ]%
Algorithmic Trading and Market Microstructure%
\item[Reference Solution: ]%
To calculate the net daily alpha after accounting for transaction costs and slippage, we need to first determine the total costs associated with the trading strategy. There are two types of costs to consider: the bid{-}ask spread and slippage. The bid{-}ask spread represents the difference between the highest price a buyer is willing to pay (bid) and the lowest price a seller is willing to accept (ask). Slippage is the difference between the expected price of a trade and the actual price at which the trade is executed.\textbackslash{}n\textbackslash{}nFirst, let's calculate the total bid{-}ask spread cost. The average bid{-}ask spread is 0.5 basis points, and the strategy generates 100 trades per day. Therefore, the total bid{-}ask spread cost per day is:\textbackslash{}n\textbackslash{}nTotal bid{-}ask spread cost = Average bid{-}ask spread * Number of trades\textbackslash{}nTotal bid{-}ask spread cost = 0.5 basis points * 100 trades = 50 basis points\textbackslash{}n\textbackslash{}nNext, let's calculate the total slippage cost. The average slippage per trade is 0.2 basis points, and the strategy generates 100 trades per day. Therefore, the total slippage cost per day is:\textbackslash{}n\textbackslash{}nTotal slippage cost = Average slippage per trade * Number of trades\textbackslash{}nTotal slippage cost = 0.2 basis points * 100 trades = 20 basis points\textbackslash{}n\textbackslash{}nNow, we can calculate the net daily alpha by subtracting the total costs from the expected daily gross alpha:\textbackslash{}n\textbackslash{}nNet daily alpha = Expected daily gross alpha {-} (Total bid{-}ask spread cost + Total slippage cost)\textbackslash{}nNet daily alpha = 5 basis points {-} (50 basis points + 20 basis points) = {-}65 basis points\textbackslash{}n\textbackslash{}nThe net daily alpha after accounting for transaction costs and slippage is {-}65 basis points. This indicates that the trading strategy is not profitable after considering the costs associated with trading.\textbackslash{}n\textbackslash{}nMarket impact models are important for assessing the performance of a trading strategy because they help quantify the impact of trading activities on the market. In this case, the market impact models would help determine how the strategy's trading activities affect the bid{-}ask spread and slippage. By incorporating market impact models into the strategy, it may be possible to reduce the costs associated with trading and improve the net daily alpha. This could involve reducing the number of trades, adjusting the holding period, or optimizing the order execution strategy to minimize the market impact.%
\item[Grading Rubric: ]%
\{'rubric': {[}\{'Question': 'Question Number 1', 'Rubric': {[}\{'Criteria': 'Calculation of total bid{-}ask spread cost', '1 Point': 'Incorrect calculation or no attempt made', '2 Points': 'Correct formula but incorrect values used', '3 Points': 'Correctly calculated total bid{-}ask spread cost'\}, \{'Criteria': 'Calculation of total slippage cost', '1 Point': 'Incorrect calculation or no attempt made', '2 Points': 'Correct formula but incorrect values used', '3 Points': 'Correctly calculated total slippage cost'\}, \{'Criteria': 'Calculation of net daily alpha', '1 Point': 'Incorrect calculation or no attempt made', '2 Points': 'Correct formula but incorrect values used', '3 Points': 'Correctly calculated net daily alpha'\}, \{'Criteria': 'Discussion of market impact models', '1 Point': 'No discussion or incorrect understanding of market impact models', '2 Points': 'Basic understanding of market impact models but lacking depth', '3 Points': 'Comprehensive discussion of market impact models and their impact on strategy performance'\}, \{'Criteria': 'Overall structure and clarity of thought process', '1 Point': 'Poor structure and unclear thought process', '2 Points': 'Good structure but some parts unclear or not well{-}explained', '3 Points': 'Clear and well{-}structured thought process throughout the answer'\}{]}\}{]}\}%
\item[Feedback on Candidate Solution: ]%
The candidate did not provide any solution to the problem. There was no attempt to calculate the total bid{-}ask spread cost, total slippage cost, or net daily alpha. Additionally, there was no discussion of market impact models, and the overall structure and clarity of the thought process were nonexistent.%
\item[Candidate Score: ]%
0 / 15%
\end{description}

%
\end{document}